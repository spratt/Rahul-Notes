\documentclass[letterpaper,12pt,twocolumn]{article}

\usepackage{savetrees}
\usepackage{xspace}
\usepackage{amssymb}
\usepackage{amsthm}
\usepackage{amsmath}

\DeclareMathOperator{\splt}{split}
\DeclareMathOperator{\range}{range}

\newcommand\etal{\textit{et al.}\xspace}

\newcommand{\BigOh}[1]{O\!\left(#1\right)}
\newcommand\IR{\mathbb{R}}
\newcommand\bounds[1]{(#1]} %chktex 9

\theoremstyle{plain}
\newtheorem{theorem}{Theorem}

\begin{document}

{\noindent\Large Notes on Saladi 2015}\\
{\noindent\large By Simon Pratt}

The primary intention of these notes is to answer: how
does~\cite{saladi2015improved} answer 3D 5-sided point enclosure
queries?

\paragraph{Orthogonal Point Enclosure Queries (OPEQ)}
Preprocess a set $S$ of $n$ axes-parallel rectangles in $\IR^3$ in
order to determine all rectangles containing a query point $q$.  In
particular, we wish to consider the case in which a rectangle is
5-sided.  To be precise, our rectangles are of the form
$\bounds{x_1,x_2} \times \bounds{y_1, y_2} \times \bounds{-\infty,z}$.

\section{Simple and Slow in Linear Space}

In this section, we will prove the following:

\begin{theorem}[2.1 in~\cite{saladi2015improved}]

  OPEQ on 5-sided rectangles can be answered using a structure of
  $\BigOh{n}$ size and $\BigOh{\log^3 n + k}$ query time, where $k$
  is the size of the output.

\end{theorem}

An \emph{interval tree} is a binary tree whose leaves store the values
of the endpoints of a set of intervals~\cite{edelsbrunner1983new}.  At
each node $v$, store $\splt(v)$ which is the largest value in the left
subtree of $v$, and $\range(v)$ which is $(-\infty,\infty)$ at the
root, and otherwise if $\range(v) = [x_\ell, x_r]$, then $v$'s left
child will have range $[x_\ell, \splt(v)]$, and symmetrically for the
right child.  An interval is assigned to the node $v$ of minimal
height such that the interval is contained within $\range(v)$.

If we additionally maintain a list $IT_v^\ell$ (resp. $IT_v^r$) that
stores the left (resp.\ right) endpoints of all intervals stored at $v$
in non-decreasing (resp.\ non-increasing) order, then space remains
$\BigOh{n}$, but we can answer a 1D OPEQ in $\BigOh{\log n + k}$ time.

\section{Faster in Near-Linear Space}

We can speed up the query time while nearly maintaining the space by
using a grid technique from~\cite{alstrup2000new} and the following
theorem on 4-sided rectangles of the form 
$\bounds{x_1,x_2} \times \bounds{-\infty, y} \times \bounds{-\infty,z}$:

\begin{theorem}[3.1 in~\cite{saladi2015improved}]

  OPEQ on 4-sided rectangles can be answered using a structure of
  $\BigOh{n\log^* n}$ size and $\BigOh{\log n \cdot \log^* n + k}$
  query time, where $k$ is the size of the output.

\end{theorem}

TODO

Which concludes the proof of the following:

\begin{theorem}[5.1 in~\cite{saladi2015improved}]

  OPEQ on 5-sided rectangles can be answered using a structure of
  $\BigOh{n\log^* n}$ size and $\BigOh{\log n \log\log n + k}$ query
  time, where $k$ is the size of the output.

\end{theorem}

\bibliographystyle{alpha}
\bibliography{Saladi-Notes-References}

\end{document} %chktex 17
